\documentclass[times, utf8, zavrsni]{fer}

\usepackage{booktabs}
\usepackage[hidelinks]{hyperref}

\begin{document}

\thesisnumber{000}
\title{Klasifikacija uporabom umjetnih neuronskih mreža}
\author{Darijo Brčina}

\maketitle

% Ispis stranice s napomenom o umetanju izvornika rada. Uklonite naredbu \izvornik ako želite izbaciti tu stranicu.
\izvornik

% Dodavanje zahvale ili prazne stranice. Ako ne želite dodati zahvalu, naredbu ostavite radi prazne stranice.
\zahvala{}

\tableofcontents

\chapter{Uvod}
Uvod rada. Nakon uvoda dolaze poglavlja u kojima se obrađuje tema.

\chapter{Pregled područja}
Pitate li se ikada što je to inteligencija te čemu nam služi. To pitanje postavljeno je još za vrijeme začetka filozofije kao znanosti kada su se tadašnji filozofi pitali kako i na koji način je ljudsko razmišljanje, učenje i pamćenje ostvareno. Ni dan danas ne postoji jednoznačan odgovor na to pitanje jer ljudski mozak i dalje predstavlja jednu veliku nepoznanicu koja vjerojatno nikada ili ne tako skoro neće biti razriješena. No znanost je podosta napredovala i shodno tomu se razvila želja da se ljudska inteligencija pokuša pretočiti u nekakvu vrstu inteligencije strojeva.

Početak ovakvog razmišljanja datira od 50-tih godina dvadesetog stoljeća kada Alan Turing u članku \textit{Computing Machinery and Intelligence} časopisa \textit{Mind} postavlja pitanje: Mogu li strojevi misliti? \engl{Can machines think?} na koje pokušava odgovoriti kroz tzv. igru imitacije \engl{imitation game}. Sudionici igre su tri igrača: igrač A, igrač B i igrač C gdje su igrači A i B ispitanici a igrač C ispitivač. Cilj igrača C je utvrditi spol ispitanika postavljanjem pitanja, cilj igrača B je pomoći ispitivaču C a cilj igrača A je navesti ispitivača C na pogrešnu identifikaciju. Što će se dogoditi ako stroj uzme mjesto igrača A? Ako broj pogrešaka igrača C bude gotovo jednak u oba slučaja, onda je stroj inteligentan. \citep{turingAI}. Ovakav princip se često naziva turingov test \engl{Turing test}.

1956. se održava konferencija u Dartmouthu \citep{wiki:DART} na inicijativu John McCarthy-a, tadašnjeg mladog profesora matematike na fakultetu u Dartmouthu, koji okuplja oko sebe nekolicinu znanstvenika i prijatelja kako bi pokušali koncepte ljudske inteligencije preslikati u inteligenciju strojeva. Cilj je pokazati kako strojevi koriste jezik, kako zaključuju i stvaraju apstraktne koncepte i kako vremenom postaju sve prilagodljiviji na predočene probleme baš kao i ljudi. Inicijalna ideja je bila da se neki od navedenih problema može dokazati uz manju skupinu dobrih znanstvenika i kroz period od jednog ljeta (McCarthy et al. 1955). To naravno nije bilo moguće. Time se formalno uvodi pojam \textit{umjetna inteligencija}.

\section{Umjetna inteligencija}
Kao što je anticipirano ranije, umjetna inteligencija \engl{artificial intelligence} je laički rečeno inteligencija strojeva a znanost koja se jednim dijelom bavi proučavanjem umjetne inteligencije jest računarska znanost \engl{computer science}.

Primjena umjetne inteligencije danas je izrazito rasprostranjena kroz gotovo svaku industriju. Pronalazimo ju u medicini, automobilskoj industriji, robotici, pa i u sportskoj i industriji igara. Jedan od poznatijih događaja koji prikazuje primjenu umjetne inteligencije dogodio se 2016. kada je računalo naziva \textit{AlphaGo} u igri \textit{Go} uspijelo pobijediti svjetskog prvaka Lee Sedolu rezultatom 4:1 te time ostvario velik uspjeh u svijetu umjetne inteligencije kao i pažnju javnosti \citep{moyerGO}.

Umjetnu inteligenciju je dakako potrebno trenirati i učiti pa je tako učenje podijeljeno na dvije veće cjeline:
\begin{center}
    \begin{enumerate}
        \item Simboličko učenje
        \item Strojno učenje.
    \end{enumerate}
\end{center}

\subsection{Simboličko učenje}
Simbolička umjetna inteligencija \engl{symbolic artificial intelligence} je izraz koji definira skup istraživačkih metoda koje se temelje na ljudima lako čitljivim simbolima \engl{human-readable simbol} koji modeliraju probleme i logiku. Jedan od najboljih primjera jesu \textit{ekspertni sustavi} koji se temelje na skupu pravila. Pravila su modelirana na sličan način kao i Ako-Onda rečenica \engl{If-Then statement} koja je u ljudskoj komunikaciji svakodnevno u upotrebi. Također, razne vrste logika poput propozicijska \engl{Propositional logic}, često referirana kao Boolova algebra, logika prvog reda \engl{First order logic}, poznatija kao predikatna logika \engl{Predicate logic}, neizrazita logika \engl{Fuzzy logic} pripadaju upravo simboličkoj umjetnoj inteligenciji. Ovakav način učenja bio je popularan početkom 1950. sve do kraja 1980 \citep{wiki:SIMB}.

\subsection{Strojno učenje}
Strojno učenje \engl{Machine learning, ML} predstavlja niz metoda i algoritama koji sustavima pružaju stjecanje novog znanja kroz modeliranje obrazaca koje onda kasnije mogu iskoristiti za predviđanje novih podataka ili sličnih \citep{cupicML}. Glavna ideja je da sustavi uče iz iskustva, empirijski, bez da se programska implementacija mijenja što znatno olakšava manipulaciju istih.

Danas postoji nezgrapno puno podataka koje je moguće i koje je potrebno iskoristiti za učenje pa je cilj konstruirati sustave koji mogu iskoristiti baš te podatke za neka korisna ponašanja poput predviđanja i raspoznavanja raznih uzoraka \citep{cupicML}.

\chapter{Zaključak}
Zaključak.

\bibliography{literatura}
\bibliographystyle{fer}
\nocite{*}

\begin{sazetak}
Sažetak na hrvatskom jeziku.

\kljucnerijeci{Ključne riječi, odvojene zarezima.}
\end{sazetak}

\engtitle{Classification Based on Artificial Neural Networks}
\begin{abstract}
Abstract.

\keywords{Keywords.}
\end{abstract}

\end{document}
