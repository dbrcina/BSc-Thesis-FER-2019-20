\documentclass[times, utf8, zavrsni]{fer}

\usepackage{booktabs}
\usepackage[hidelinks]{hyperref}

\begin{document}

\thesisnumber{000}
\title{Klasifikacija uporabom umjetnih neuronskih mreža}
\author{Darijo Brčina}

\maketitle

% Ispis stranice s napomenom o umetanju izvornika rada. Uklonite naredbu \izvornik ako želite izbaciti tu stranicu.
\izvornik

% Dodavanje zahvale ili prazne stranice. Ako ne želite dodati zahvalu, naredbu ostavite radi prazne stranice.
\zahvala{}

\tableofcontents

\chapter{Uvod}
Uvod rada. Nakon uvoda dolaze poglavlja u kojima se obrađuje tema.

\chapter{Pregled područja}
Pitate li se ikada što je to inteligencija te čemu nam služi, nama ljudima. To pitanje postavljeno je još za vrijeme začetka filozofije kao znanosti kada su se tadašnji filozofi pitali kako i na koji način je ljudsko razmišljanje, učenje i pamćenje ostvareno. Ni dan danas ne postoji jednoznačan odgovor na to pitanje jer ljudski mozak i dalje predstavlja jednu veliku nepoznanicu koja vjerojatno nikada ili ne tako skoro neće biti razriješena. No znanost je podosta napredovala i shodno tomu se razvila želja da se ljudska inteligencija pokuša pretočiti u nekakvu vrstu inteligencije strojeva.

Početak ovakvog razmišljanja datira od 50-tih godina dvadesetog stoljeća kada Alan Turing u članku \textit{Computing Machinery and Intelligence} časopisa \textit{Mind} postavlja pitanje: Mogu li strojevi misliti? \engl{Can machines think?} na koje pokušava odgovoriti kroz tzv. igru imitacije \engl{imitation game}. Sudionici igre su tri osobe: igrač A, igrač B i igrač C gdje su igrači A i B ispitanici a igrač C ispitivač. Cilj igrača C je utvrditi spol ispitanika postavljanjem pitanja, cilj igrača B je pomoći ispitivaču C a cilj igrača A je navesti ispitivača C na pogrešnu identifikaciju. Što će se dogoditi ako stroj uzme mjesto igrača A? Ako broj pogrešaka igrača C bude gotovo jednak u oba slučaja, onda je stroj inteligentan \citep{turingAI}.

1956. se održava konferencija u Dartmouthu \citep{wiki:DART} na inicijativu John McCarthy-a, tadašnjeg mladog profesora matematike na fakultetu u Dartmouthu, koji okuplja oko sebe nekolicinu znanstvenika i prijatelja kako bi pokušali koncepte ljudske inteligencije preslikati u inteligenciju strojeva. Cilj je pokazati kako strojevi koriste jezik, kako zaključuju i stvaraju apstraktne koncepte i kako vremenom postaju sve prilagodljiviji na predočene probleme baš kao i ljudi. Inicijalna ideja je bila da se neki od navedenih problema može dokazati s nekolicinom dobrih znanstvenika i kroz period od jednog ljeta (McCarthy et al. 1955). To naravno nije bilo moguće. Time se formalno uvodi pojam \textit{umjetna inteligencija}.

\section{Umjetna inteligencija}
Kao što je anticipirano ranije, umjetna inteligencija \engl{artificial intelligence} je laički rečeno inteligencija strojeva a znanost koja se jednim dijelom bavi proučavanjem umjetne inteligencije jest računarska znanost \engl{computer science}. Glavna ideja je da se većina

Primjena umjetne inteligencije danas je izrazito rasprostranjena kroz gotovo svaku industriju. Pronalazimo ju u medicini, automobilskoj industriji, robotici, pa i u sportskoj i industriji igara. Jedan od značajnijih primjera primjene umjetne inteligencije je

\chapter{Zaključak}
Zaključak.

\bibliography{literatura}
\bibliographystyle{fer}
\nocite{*}

\begin{sazetak}
Sažetak na hrvatskom jeziku.

\kljucnerijeci{Ključne riječi, odvojene zarezima.}
\end{sazetak}

\engtitle{Classification Based on Artificial Neural Networks}
\begin{abstract}
Abstract.

\keywords{Keywords.}
\end{abstract}

\end{document}
